% ************************** Thesis Abstract *****************************
% Use `abstract' as an option in the document class to print only the titlepage and the abstract.
\begin{abstract}
Recent advances in high-power electronic devices using wide bandgap materials such as SiC and GaN highlight a growing need for enhanced efficiency, longevity, and capacity in energy applications. As silicon-based IGBTs, crucial in global power conversion, face challenges in achieving high efficiencies, the urgency to reduce energy loss becomes more pronounced amidst rising climate concerns and increasing global energy demand. Diamond, renowned for its exceptional material properties and figures of merit, stands on the cusp of revolutionising power electronics, potentially marking the beginning of a new era in high-performance devices. Despite diamond's potential, significant foundational work in doping, metal contact formation, and crystal growth is essential for its practical deployment. While p-type boron-doped diamond has matured, n-type doping with phosphorous continues to be an active research area. Conceptual devices using both n-type and p-type diamond show promise, yet practical limitations of diamond such as that of the strong Fermi-level pinning experienced necessitate innovative manufacturing approaches. This work focuses on the intricacies of fabricating ohmic contacts on nominally phosphorous-doped diamond, supported by thorough material characterisation, to advance the development of feasible diamond-based electronic devices.

A notable development reported here is the enhancement of ohmic contacts through laser graphitisation, a method that significantly improves the electrical performance of contacts to phosphorous-doped diamond. This paves the way for more efficient diamond-based devices, wherever ohmic contacts are required. The mechanisms behind enhanced ohmic contacts are also investigated, with potential device applications examined via finite element modelling. These findings paint an encouraging picture for diamond-based technologies in sectors where energy efficiency is paramount, aligning with the global pursuit of sustainability. As the production of large-scale diamond substrates progresses, diamond-based devices employing innovative techniques like laser graphitisation are well-positioned to penetrate high-power system markets, especially in extreme environments where silicon fails, and even other wide bandgap materials like SiC may falter. Particularly promising are devices like solid-state cold cathodes that leverage diamond's negative electron affinity surface terminations, finding applications in Hall effect thrusters and X-ray sources. Devices reliant on p-n or p-i-n junctions, such as in LED applications, will also benefit from the advancements in reliable ohmic contacts. Similarly, the development of improved Schottky diodes opens avenues for higher performance and broader application.
\end{abstract}
