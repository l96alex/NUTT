\begin{refsection}
\chapter{Experimental Work 2} % better naming required since this is the second experimental chapter now
\section{Introduction}
\subsubsection{Laser Writing Thickness}
\label{subsubsec:laser_writing_thickness}
One final detail is necessary to consider the effective resistivity of the laser processed surface wires -- that of the thickness of these processed regions. A full overview of the laser processing technique is outlined more fully in chapter \ref{ch:laser}, however the salient point regarding laser treated thickness can be estimated based on key studies in which the structure of laser processed diamond samples are examined via techniques such as TEM \cite{salter2017}, SEM \cite{ashikkalieva2022} and interference based microscopy techniques following the removal via oxidation of graphitic material \cite{kononenko2005}. Despite the topological complications induced by ablation during the laser processing technique, it is reasonable to estimate based on this prior work that the thickness of material in which a graphitisation or general sp$^{3}$ bond breaking process has occurred is likely to be relatively continuous, with only small abnormalities on average across the laser treated region. 

With consideration of the literature that is comparable to the laser processing performed to create these device structures, a simple assumption that will allow for further analysis of the electrical properties of laser written wires is to take an estimated thickness of 500~\si{\nano\metre} for the effective thickness. This then represents material that may be considered to contain graphitised phases of carbon. This is consistent with the photographitisation/thermal graphitisation process that is described in chapter \ref{ch:laser}, though it must be stated that the exact thickness of the graphitised material will be a function of position along the wire, and it is unlikely to be uniform in reality. 

132\si{\electronvolt^{-3/2}\volt\per\nano\metre}

$\sim43534$


\printbibliography[heading=subbibliography]
\end{refsection}

