\begin{refsection}
\chapter{Conclusions and Future Work}
\label{ch:conclusions}
\section{Overview}
In this thesis, a wide range of experimental techniques, combined with computational simulations, have been performed to study the practicality and possible improvement of diamond electronic devices. In particular, the usage of phosphorous doped diamond has been a key interest, due to the significant number of applications that n-type diamond has great promise in. Boron doped, p-type diamond is relatively well established in 2024, but the continual improvement, and occasional leap forward in diamond growth techniques hints at a future in which diamond devices utilising phosphorous doping are not just designed on the back of an envelope. Direct laser writing of graphitic, or more recently discovered diaphitic allotropes of carbon within diamond substrates may represent a substantial improvement in the formation of phosphorous doped diamond devices. To briefly summarise the work presented in the previous chapters:

\section{Laser writing of phosphorous doped diamond significantly reduces contact resistance}

\subsection{Summary of metal contacts}
The literature best-case value for the specific contact resistivity with titanium contacts is $10^{-3}$~\si{\ohm\centi\metre\squared} \cite{matsumoto2013}. For the LTLM samples in this thesis, specific contact resistivity was measured to be 472 and 494\si{\kilo\ohm\centi\metre\squared}, with a resistivity of the phosphorous doped film of 132 and 464~\si{\mega\ohm\centi\metre} for samples C and D respectively at 10 V. For sample F and the CTLM results, at a constant current condition of -100 \si{\nano\ampere}, a specific contact resistivity of around 2.1~\si{\kilo\ohm\centi\metre\squared} and resistivity of around 490~\si{\kilo\ohm\centi\metre} was measured. Following these observations, it was clear that an alternative approach must be taken to form superior ohmic contacts on these samples which are sufficient to allow for competitiveness between diamond based power devices and current best case values on standard materials. For example, SiC is able to achive extremely low resistance ohmic contacts below $1\times10^{-7}$~\si{\ohm\centi\metre\squared} \cite{pan2013}. 

\subsection{Laser Writing Summary for LTLM Specific Contact Resistivity Reduction}
In section \ref{subsec:ltlm2_lost_data}, it is established that by using a LTLM setup comparable to that of the first two samples tested, a specific contact resistance of 1.62~\si{\ohm\centi\metre\squared} is obtained, with an observed phosphorous doped resistivity of 6~\si{\kilo\ohm\centi\metre}. This is a drastic drop of 5 orders of magnitude in the specific contact resistance, matching the current best thermal graphitisation and coaxial arc plasma deposition nanocarbon contacts.

\section{Emitter Structures}
Emitter structures, as tested, have produced a strong asymmetric current. The practicality of these device structures remains speculative, in part due to the transient nature of these emitters. The clear strengths of laser writing for the purpose of tailoring the allotropes of carbon within solid blocks of diamond are on full display. In this case, it has made phosphorous doped diamond which may not be suitable for conventional bulk semiconductor type devices into a potential emitter substrate. 

\section{Future Work}
It is the opinion of this student that laser writing for the purpose of generating a wide range of practical power electronic applications is on the near horizon. With the ability to utilise n-type diamond with a single, simple processing stage, countless studies into the etching of diamond for the sake of specific contact reduction, and the generation of mesa steps or sharp emitters can be called into question. Instead, one only needs to design a 3D structure, much like the current generations of 3D transistor GAAFET technology, and then they can print it within a diamond directly. Diamond power electronics that take advantage of the possibility of not only writing conductive wires between n or p type doped regions of diamond, but also writing semiconducting phases directly within the diamond bulk open pandoras box. Perhaps diamond over the next century will develop from a niche, challenging material, into the must have semiconductor, especially with brand new methods of growing diamond like liquid metal diamond growth appearing quite by surprise in the last weeks of thesis writing.

\printbibliography[heading=subbibliography]

\end{refsection}