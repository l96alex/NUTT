\begin{refsection}
\begin{introduction}
\chaptermark{Introduction}
\begin{section}{Background}
{
Electronic devices based on transistors have transformed the world, with common everyday devices such as smartphones and computers dependent upon a vast infrastructure of modern engineering. One particular focus in the modern world is on the usage of energy, and how best to reduce the amount of energy that we consume. This is an essential step as we progress towards a world in which renewable energy sources can sustain all of the normal devices that we have come to rely on. Power electronics is a unique bottleneck for electrical power consumption, as power electronic devices such as diodes, thyristors, and transistors in general are crucial for all applications. Just as micro-electronics pushes ever onwards with the miniaturisation of small transistors for the purpose of manufacturing processors with increasing densities, power electronics endeavours to provide the electrical energy necessary for these devices. 

Unlike micro-electronics, which has found ingenious ways to maintain progress in miniaturisation through continual refinement of the manufacturing processes, power electronics is fundamentally limited by the materials at the heart of such devices. Alternative semiconductors to that of silicon are fast developing, with examples such as that of silicon carbide and gallium nitride clearly demonstrating the practical applications of devices based on wide bandgap semiconductors. However progress at the top-end of power electronics is relatively static, with devices over the last decade or more rarely showing any substantial improvements, and still being generally reliant upon silicon. One competitor which may end the reign of silicon in power electronics is that of diamond. 

In the last 20--30 years, massive technological progress has been achieved in the growth, doping and surface treatment of diamond. Many varieties of diamond-based devices have been tested and proven in a prototypical fashion, such as Schottky diodes, field effect transistors, bipolar transistors, metal oxide based devices and even light emitting diode applications. In particular, laser treatment of diamond has come to the forefront of diamond device manufacturing processes, with significant implications for the fabrication of devices that may be able to make use of the offered ability to laser write devices made up of graphitic, diaphitic \cite{Nemeth2021}, and diamond forms of carbon. One may speculate on a future in which power electronic devices are based entirely upon carbon, with diamond offering both a substrate for the fabrication of devices, as well as the best ability to act as a heatsink of any material. However, to achieve this, further development in the field of diamond electronics is required. With new methods of diamond growth at room pressure and utilising liquid metal catalysts still being discovered \cite{Gong2024}, the future of diamond specific techniques remains bright.
}
\end{section}
\begin{section}{Aims and Objectives}
{
 In this project, initial aims primarily focused upon the development and refinement of devices based upon cold field effect emitters within diamond. However, the challenges of fabricating diamond electronics with phosphorous doped diamond led to work being pursued in the methodologies of diamond device fabrication in a more general context. Of particular interest was the plausibility of laser writing devices into diamond, which offers both simplification of the fabrication processes and potentially an improvement when compared to the novel, diamond unique techniques actively being developed. One specific hypothesis that was tested was the comparability of these laser graphitisation techniques
}
\end{section}
\begin{section}{Thesis Outline}
{
This thesis contributes to the ongoing worldwide research into diamond electronics by investigating the electrical performance of phosphorous doped diamond and the formation of electrical contacts to said diamond. Additionally, the concept of forming devices which take advantage of diamond to form field effect emitters is examined, with laser written test structures demonstrating that these devices may well have some merit for further investigation. The thesis is arranged as follows:

\paragraph{Chapter 1 - Introductory Content and Historical Footing}
This chapter attempts to explain the significance of diamond in electronic applications through the lens of historical development at large in the field of electrical devices. 

\paragraph{Chapter 2 - Theory and Background}
In this chapter, the specific diamond properties which make it such an attractive prospect for power electronics are introduced. Further to this, fundamental theoretical work that is essential for later work within the thesis is outlined to provide a general reference before experimental work is presented in the following chapters.

\paragraph{Chapter 3 - Metal Contacts to Phosphorous Doped Diamond}
With this chapter, the first sets of experimental work concerning the usage of metal contacts to phosphorous doped diamond are presented. While the initial goal of such experiments was to characterise the phosphorous doped diamond itself, significant challenges due to low electrical conduction through these samples led to investigations into how these simple devices may be improved. With phosphorous doped diamond still a relatively rare substrate due to the challenges involved with growth of high quality doped diamond, effort was taken to maximise the work that can be done on only a handful of samples.

\paragraph{Chapter 4 - Characterisation of Laser Processed Phosphorous Doped Diamond}
Following the work performed with metal contacts formed on phosphorous doped diamond, laser processing is investigated as a potential avenue for improvement of electrical contacts and also for specific device structure formation. This chapter contains the characterisation work that was performed on a laser processed sample, in an effort to better understand the resulting electrical behaviours.

\paragraph{Chapter 5 - Laser Processing for Ohmic Contacts}
In this chapter, the laser processing is experimentally tested to compare and contrast with the earlier metal contact based devices. Within the backdrop of novel approaches to forming ohmic contacts to phosphorous doped diamond, the data presented herein represent a new method of reducing contact resistance. As part of this, more specific characterisation is performed of the laser written structures themselves, to provide a suitable backing for the more complex device structures that make use of phosphorous doped diamond channels.

\paragraph{Chapter 6 - Testing of Laser Written Emitters and Simulations}
With this chapter, electrical characterisation of novel laser written emitter structures is outlined. To elucidate on the physical meaning behind the observed results of experimentation, computational modelling is used to further investigate these devices.


\paragraph{Chapter 7 - Conclusions and Future Work}
Finally, this chapter summarises the key findings of the work reported in this thesis, and sets out the next steps that may be taken to further develop diamond electronic devices that make use of laser processing and phosphorous doped diamond. These two components may be critical to the future commercial success of diamond power electronics.
}
\end{section}

\end{introduction}

\printbibliography[heading=subbibliography]

\end{refsection}