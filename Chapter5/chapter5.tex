\chapter{Computational Modelling of Diamond-Based Devices}

\section{Introduction}
The emergent field of diamond-based electronics has prompted an exploration into the electrostatic and semiconductor modelling to predict device behavior accurately, forming the cornerstone of this computational chapter.

\section{Electrostatic Modelling of Cold Field Effect Emitters}
Electrostatic phenomena at the microscale govern the performance of cold field effect emitters, necessitating a detailed computational approach to model these effects.

\subsection{Geometry and Electrostatic Profiling}
The geometry of cold cathodes, particularly the tip sharpness and surface termination, significantly influences the electrostatic profile, which has been systematically analyzed through finite element methods.

\subsection{Influence of Cathode-Anode Voltages}
Variations in cathode-anode voltages alter the electric field distribution, revealing a complex relationship between applied potential and field enhancement factors.

\subsection{Gating Structures and Field Density}
The incorporation of gating structures into the emitter design was modeled to assess its impact on the control of electron emission and field density at the emitter apex.

\section{Semiconductor Modelling}
This section delves into the semiconductor properties of diamond-based devices, where carrier mobility and interface barriers play pivotal roles in device behavior and performance.

\subsection{Carrier Mobility Modelling}
In addressing the challenge of accurately modelling carrier mobility, we adopted a pragmatic approach that leverages empirical relationships derived from experimental hall mobility measurements, providing a synthesized reflection of various scattering mechanisms.

\subsection{Schottky Barrier Modelling}
The Schottky barrier, a critical feature at the metal-semiconductor interface, was modelled to quantify its impact on device operation, employing both theoretical constructs and experimental insights to establish a comprehensive profile.

\subsection{Activation of Phosphorous Donors}
The activation of phosphorous donors within diamond substrates, a critical aspect of device functionality, has been analyzed, questioning traditional assumptions in high-donor-concentration scenarios.

\subsection{Thermal Ionisation and Electric Field Ionisation}
Our models suggest that the thermal ionisation of phosphorous donors remains the primary mechanism of carrier generation, with electric field ionisation playing a potentially lesser role than previously anticipated.

\section{Transfer Length Method (TLM) Experiment Modelling}
The TLM experiments served as a complementary approach to validate our computational models, providing empirical data against which the simulations were benchmarked.

\subsection{Metal Contact Patterning and Schottky Barriers}
The patterning of metal contacts on diamond substrates introduces varied Schottky barriers, influencing current flow and necessitating detailed modelling to predict device behavior.

\subsection{Current-Voltage Characterisation}
The current-voltage characteristics derived from computational models have offered insights into the electrical properties of the contacts, informing the design of TLM experiments.

\section{Further Device Modelling}
Beyond the initial models, the scope of computational analysis is expanded to encompass a broader range of diamond-based devices and their novel features.

\subsection{Vertical Schottky Barrier Diodes}
Preliminary models of vertical Schottky barrier diodes are discussed, providing a foundational understanding of their potential in high-power applications.

\subsection{Floating Guard Ring Diodes}
The design and modelling of floating guard ring diodes address the challenges of edge breakdown, a critical consideration for device reliability.

\subsection{Laser Graphitisation Effects}
Laser graphitisation introduces localized conductivity changes within diamond; modelling these effects helps predict the implications for device performance and fabrication.

\section{Discussion}
This discussion synthesizes the modelling outcomes, drawing correlations between the computational predictions and the experimental data, and laying out the path for future research directions.

\section{Future Work and Device Modelling Expansion}
While the scope of this thesis is defined by the computational work completed to date, the potential for further exploration in diamond-based device modelling is vast. The ambition to expand upon the current models is tempered by the computational challenges historically encountered, particularly when integrating the semiconductor module and Schottky barriers within COMSOL.

\subsection{Diamond as a Thermal Conductor and Heat Sink Applications}
The exceptional thermal conductivity of diamond makes it a candidate of significant interest for advanced heat sink applications. This section explores the potential of leveraging diamond coatings to enhance thermal management in high-performance electronic devices.

\subsubsection{Thermal Conductivity of Diamond}
An overview of the intrinsic thermal properties of diamond is presented, highlighting why diamond stands out as a superior thermal conductor compared to traditional materials.

\subsubsection{Modelling Heat Dissipation in Diamond-Coated Heat Sinks}
Computational models simulating heat dissipation in diamond-coated heat sinks are discussed. These models aim to quantify the thermal performance improvements that diamond coatings can provide in practical electronic applications.

\subsubsection{Integration with Electronic Device Modelling}
This subsection delves into how the thermal models of diamond heat sinks can be integrated with the electronic device models discussed earlier in the thesis. The goal is to present a holistic view of diamond's role in both electronic and thermal performance enhancement.

\subsubsection{Future Prospects and Challenges}
Potential future applications of diamond in thermal management are explored, along with the challenges that need to be addressed to realize these applications. This includes considerations of diamond coating processes, interface thermal resistance, and integration with various electronic device architectures.

\subsection{Challenges with Schottky Barriers in COMSOL}
Past experiences with non-converging models in the presence of Schottky barriers have instilled a cautious approach to simulations. A reflection on these challenges and the strategies employed to mitigate them reveals the intricate balance between model complexity and computational feasibility.

\subsection{Prospects for Modelling Additional Devices}
As my proficiency with COMSOL/FEM and understanding of diamond-based electronics have grown, so too has the potential to rapidly prototype and simulate new device structures. While such ambitions are moderated by the focus required for essential models, there remains a keen interest in extending the computational analysis to include other promising device geometries in the future.

\section{Conclusion}
The computational exploration documented in this chapter provides a substantive contribution to the field of diamond electronics, offering predictive insights and guiding experimental endeavors.
