% Chapter 2

% Chapter X

\chapter{Review of Diamond and Diamond for Technology} % Chapter title

\label{ch:review_of_diamond} % For referencing the chapter elsewhere, use \autoref{ch:name} 

%----------------------------------------------------------------------------------------
\part{Semiconductor Physics and Diamond}
\section{Introduction}

%This chapter will be used as a set of notes from Yu 2010, perhaps with a blending of Sze and Schroder. Yes I should have written this 3 years ago. I know. Let's get a good 2k words of it done this weekend and have it in my pocket for more advanced writing such as the metal-semiconductor chapter and graphitisation of diamond etc...
Semiconductors can be defined in several ways, all of which are correct, depending on the context. For example, they could be defined based on the resistivity range of typical metals. Metals such as aluminium, caesium, copper, gold, iron, molybdenum, platinum and silver have resisitivities in the range of $1.62\times10^{-8}\si{\ohm\metre}\text{--}20.5\times10^{-8}\si{\ohm\metre}$ at room temperature, as shown in table \ref{tab:metal_resisitivities}. Slightly atypical metals include bismuth and gadolinium, with an order of magnitude greater resistivity. Semiconductors could then be defined as being in the range of around $10^{-5}\text{--}10^{7}\si{\ohm\metre}$, as having a significant resistance while still providing a conduction pathway.

\begin{table}
\begin{tabular}{|l|l|l|l|l|l|l|l|l|l|l|}
\hline
Metallic Element            & Al   & Cs   & Cu   & Au   & Fe   & Mo   & Pt   & Ag   & Bi  & Gd  \\ \hline
$\rho \left(10^{-8}\si{\ohm\metre}\right)$ 273--300 K & 2.71 & 20.5 & 1.72 & 2.21 & 9.87 & 5.47 & 10.5 & 1.62 & 107 & 131 \\ \hline
\end{tabular}
\caption{Some typical metal resisitivities at room temperature with the values for Bismuth and Gadolinium from the CRC Handbook of Chemistry and Physics 97th edition \cite{haynes:2017}.}
\label{tab:metal_resisitivities}
\end{table}

They can also be defined as a material in which the highest occupied energy band is completely filled with electrons at zero temperature, with an energy gap to the next highest band ranging from 0 to $4\text{--}5\si{\electronvolt}$. With this definition, at increasing temperatures electrons are excited from the highest occupied band (valence band) to the lowest unoccupied band (conduction band), increasing conductivity or reducing resistivity, in direct contrast to the behaviour of metals. However the commonly accepted bandgap range may vary from source to source, with diamond regularly being presented as a semiconductor despite its wide bandgap of $5.47\si{\electronvolt}$ \cite{wort:2008}.

\subsection{A Survey of Semiconductors}
\subsubsection{Elemental}
The most ubiquitous semiconductor is naturally silicon, which forms a 'diamond', cubic tetrahedral crystal structure with a lattice parameter of $0.543\si{\nano\metre}$ and nearest neighbour distance of $0.235\si{\nano\metre}$ \cite{haynes:2017}. A few other key properties are  The unit cell of a diamond cubic crystal structure has a face centred cubic (fcc) lattice with a basis of two silicon atoms. This is depicted in figure \ref{fig:silicon_unit_cell}.
\begin{figure}
\centering
\begin{subfigure}[b]{0.45\textwidth}
    \centering
    \includegraphics[width=\textwidth]{gfx/Silicon_unit_cell.png}
    \caption{}
    \label{fig:silicon_unit_cell}
\end{subfigure}
\hfill
\begin{subfigure}[b]{0.45\textwidth}
    \centering
    \includegraphics[width=\textwidth]{gfx/periodic_table-black_cropped.png}
    \caption{}
    \label{fig:periodic_table_groupIV_crop}
\end{subfigure}
\caption{(a) The diamond unit cell for silicon and (b) the elemental semiconductors in group IVA with neighbouring elements.}
\end{figure}

Germanium and carbon also have a diamond crystal structure, each atom has four-fold coordination, forming the tetrahedron shape. Some other elements may also be considered semiconductors, such as phosphorus, sulphur, selenium and tellurium. These form three-fold (P), two-fold (S, Se, Te) and four-fold coordinated structures, the differing crystal structures allowing for the formation of glass.

\begin{figure}
    \centering
    \includegraphics[width=0.45\textwidth]{gfx/GaAs_structure.png}
    \caption{The crystal structure of GaAs}
    \label{fig:GaAs_unit_cell}
\end{figure}

\subsubsection{Binary Compounds}
As opposed to the single elements typically from group IVA, it is also possible to join elements from either side (group III and V), resulting in compounds that have similar properties to that of a group IV element. A good example of this is gallium arsenide (GaAs), which has a quasi cubic-close packed array of ions (arsenic) with another set of ions (gallium) residing within tetrahedral interstices between the closely packed layers. This is known as the 'zinc blende' crystal structure. The resulting lattice constant is $0.565\si{\nano\metre}$ for GaAs. As the gap in the periodic table groupings increases, the ionicity of the resulting crystal increases further and so the electronic bandgap will also increase. Hence, most of the II--VI compounds have bandgaps that are larger than 1\si{\electronvolt}. In I--VII compounds such as CuCl, the bandgaps will tend to bandgaps above 3\si{\electronvolt}. These compounds are mostly considered to be insulating materials, and not semiconductors. Two key large bandgap compound semiconductors are that of GaN and \ce{Ga_{1-x}In_{x}N} which are now widely used in blue light emitting diodes and lasers.

\subsubsection{Oxides}
Most oxides are used as effective insulators, such as \ce{SiO2}, but oxides like \ce{CuO} or \ce{Cu2O} instead are widely known to be semiconductors. The growth processes of oxide semiconductors generally limits the potential applications however. A II--VI compound of  \ce{ZnO} forms one exception, with applications as a transducer and adhesive tapes. Copper oxides have found alternative applications with the discovery of high temperature superconductivity based originally on lanthanum copper oxide (\ce{La2CuO4}).

\subsubsection{Layered}
Compound semiconductors may also be characterised by their layered crystal structures. For example, molybdenum disulfide (\ce{MoS2}) and lead iodide (\ce{PbI2}). Bonding within the layer is usually covalent, hence being significantly stronger than the van der Waals forces that stick the layers together. Electrons within these layers will have quasi-two-dimensional properties and the inter-layer interactions can be modified through the introduction of foreign atoms inbetween the layers by the process of "intercalation".

\subsubsection{Organic}
These solids are made up of carbon and hydrogen pi-bonded molecules or polymers, with heteroatoms such as oxygen, sulphur and nitrogen. These organic semiconductors will generally be in the form of molecular crystals, or amorphous thin films which are insulating, but behave as a semiconductor upon the injection of charges or through doping or photoexcitation. The first device to be built which incorporated an organic semiconductor was in 1973 by J. McGinness, which used synthetically produced melanins to act as an amorphous semiconductor threshold switch.

More recently, organic light emitting diodes (OLEDs) have become commonplace, utilising an organic compound for the emissive electroluminescent layer. Practical OLED displays are now among the thinnest of display technologies, with a higher contrast ratio due to the small scale of OLEDs allowing for the ability to turn off the backlighting for individual pixels.

\subsubsection{Magnetic}
In quite an exotic category of semiconductors, compounds that contain magnetic ions such as manganese (\ce{Mn}) and europium (\ce{Eu}) can also display both semiconducting and magnetic properties. Depending on the quantity of magnetic ions included in alloys such as \ce{EuS} \cite{asuigui2020} or \ce{Cd_{1-x}Mn_{x}Te} \cite{hossain2015}, compounds may exhibit ferromagnetism and antiferromagnetism. Alloys containing a relatively low amount of magnetic ions have applications in optical modulators, due to their large significant magneto-optical effects. 

\subsubsection{Other}
There are still yet more semiconductors that do not fit any of the previous sections such as antimony sulphoiodide \ce{SbSI} which exhibits ferroelectricity at low temperatures. Compounds of group IB-IIIA-VIB and IIV-IVA-VA such as silver gallium sulphide \ce{AgGaS2} which is useful for near or mid-infrared applications due to its large nonlinear optical coefficients \cite{rudysh2022}, zinc silicon phosphide \ce{ZnSiP2} with pressure driven emergent superconductivity \cite{yuan2021} and copper indium selenium \ce{CuInSe2} which has applications in thin film solar cell applications \cite{haloui2015}. These compounds also have tetrahedral bonding and are hence related to the IIIA-VA or IIB-VIA semiconductors of zinc blende structure.

\subsection{Wide Bandgap Semiconductors}
Briefly mentioned in the previous section, there are numerous semiconducting materials with a bandgap above $\approx2\si{\electronvolt}$ which are referred to as wide bandgap semiconductors. As the bandgap increases, the maximum operating voltage, frequency and temperature of the material will also increase, offering high performance semiconductors that can operate in more extreme conditions than that of silicon or gallium arsenide.

There are a number of wide bandgap semiconductors in usage today. For example, gallium nitride with a bandgap of $3.44\si{\electronvolt}$ was used in the first high efficiency blue LED's and lasers in the alloys of InGaN/AlGaN with an active layer of zinc doped InGaN \cite{nakamura1994}. This work utilised a two-flow metalorganic chemical vapor deposition (MOCVD) method to form the first candela class high brightness double hetero-structure blue LED's. Silicon carbide, in the 3C, 4H or 6H crystal polytypes, is again used in applications where performance at high temperature or high voltage is required.

\begin{table}
\begin{tabular}{|l|l|l|l|l|}
\hline
Properties                              & Si      & 4H-SiC    & GaAs       & GaN       \\ \hline
Crystal Structure                       & Diamond & Hexagonal & Zincblende & Wurtzite \\ \hline
Energy Gap: EG(eV)                      & 1.12    & 3.26      & 1.43       & 3.5       \\ \hline
Electron Mobility: µn(cm2/VS)           & 1400    & 900       & 8500       & 1250      \\ \hline
Hole Mobility: µp(cm2)                  & 600     & 100       & 400        & 200       \\ \hline
Breakdown Field: EB(V/cm)           & 0.3     & 3         & 0.4        & 3         \\ \hline
Thermal Conductivity(W/cm?)             & 1.5     & 4.9       & 0.5        & 1.3       \\ \hline
Saturation Drift Velocity: vs(cm/s)& 1       & 2.7       & 2          & 2.7       \\ \hline
Relative DC Dielectric Constant: eS        & 11.8    & 9.7       & 12.8       & 9.5       \\ \hline
\end{tabular}
\caption{A table to show the relative properties of silicon, silicon carbide, gallium arsenide and gallium nitride.}
\label{tab:wide_bandgap}
\end{table}
%Just a comment on dielectric constants. This constant is simply the resistance to electric fields and may vary based on dc and ac currents. When placed in a capacitor, the dielectric material is made of atoms and moleculaes and when placed inbetween the plates of a charged capacitor the negative charges in the dielectric are going to get attracted to the positive plate of the capacitor. but the negatives can't travel to the positive as the dielectric is a nonconducting material but the negatives can shift or lean towards the positive plate. this causes the charge in the atoms and the molecules within the dielectric to become polarised. to put it another way the atom kind of stretches and one end becomes overall negative and the other end becomes overall positive. it's also possible that the dielectric material started off polarised as some materials are naturally polarised like water. in this case when the dielectric is placed between the charged up capactior plates the attraction betweeen the negative side of the polarised molecule and the positive plate of the capacitor would cause the polarised molecules to rotate allowing the negatives to be a little bit closer to the positively charged capacitor plate, either way the end result is that the negatives in the atoms and molecules are going to face the positive capacitor plate and the positives in the atoms and molecules are going to face the negative capacitor plate. so how does this increase the capacitance? the reason this increases the capacitance is because this reduces the voltage between the capacitor plates. it reduces the voltage because even though there's still just as many charges on the capacitor plates their contribution to the voltage across the plates is being partially cancelled. in other words some of the positive charges on the capacitor plate are having their contribution to the voltage negated by the fact that there's a negative charge right next to them now. similarly on the negative side there's just as much negative charges there ever was but some of the negative charges are having their contribution to the voltage cancelled by the fact that there's a positive charge right next to them. so the total charge on this capacitor has remained the same but the voltage across the plates has been decreased because of the polarisation of the dielectric. if we look at the definition of capacitance we see that if the charge stays the same and the voltage decreases the capacitance is gonna increase c =q/v because divideing by a smaller number for the voltage is gonna result in a larger value for the capacitance. so, inserting a dielectric in this case increased the capacitance by lowering the voltage. let's look at another case of inserting a dielectric. the charge also increases due to more charge being induced on the plates. dielectric constant kappa k. capacitance C becomes kC where k is always >= 1. that was a bit of a waste of time but nice typing practice.
\section{Band Structures}
As noted in the introduction, a defining feature of semiconductors is the reduction of resistivity as temperatures increase. This arises due to the electronic band structure of semiconductors, which also generates the presence of gaps in their electronic excitation spectra. 

\subsection{Quantum Mechanics}
The Hamiltonian of a perfect crystal is defined as:
\begin{equation*}
\begin{aligned}
    H = \sum_{i}\frac{p_{i}^{2}}{2m_{i}} + \sum_{j}\frac{p_{j}^{2}}{2M_{j}} + \frac{1}{2}\sum_{j\prime,j}\prime\frac{Z_{j}Z_{j\prime}e^{2}}{4\pi\epsilon_{0}\left|R_{j}-R_{j\prime}\right|} - \sum_{j,i}\frac{Z_{j}e^{2}}{4\pi\epsilon_{0}\left|r_{i}-R_{j}\right|}\\ + \frac{1}{2}\sum_{i,i\prime}\prime\frac{e^{2}}{4\pi\epsilon_{0}\left|r_{i}-r_{i}\right|}
    \label{eq:hamiltonian_of_crystal_structure}
\end{aligned}
\end{equation*}
Where $\epsilon_{0}$ is the permittivity of free space, e is the electronic charge, $r_{i}$ is the position of the $i$th electron, $R_{j}$ the position of the $j$th nucleus, $Z_{j}$ represents the atomic number and $p_{i},P_{j}$ are momentum operators of the electrons and nuclei respectively. Finally, $\sum\prime$ is used to denote summation over non-identical pairs of indicies. This Hamiltonian is often simplified with the Hartree-Fock method.
%OK https://web.northeastern.edu/afeiguin/phys5870/phys5870/phys5870.html IS AMAZING..... MODERN COMPUTATIONAL METHODS IN SOLIDS
\subsubsection{Schr{\"o}dinger's Equation}
\subsubsection{Variational Methods}
\subsubsection{Examples of Linear Variational Calculations}
\paragraph{Infinite Potential Well}
\paragraph{Hydrogen Atom}
\subsubsection{Hartree-Fock Method}
%https://web.northeastern.edu/afeiguin/phys5870/phys5870/node8.html
\begin{equation}
    H = T_{N}\left(R\right) + T_{e}\left(r\right) + V_{NN}
\end{equation}

In solving the many-particle Hamiltonian, a number of approximations are required:

\begin{itemize}
    \item Valence ($i$, $i\prime$) and core ($j$, $j\prime$) electrons.
    \item Born-Oppenheimer or adiabatic approximation - stationary nucleus.
    \item Mean-field approximation - average potential $V\left(r\right)$.
    \item Also, Pauli's exclusion principle. 
\end{itemize}

\subsubsection{Valence and Core Electrons}

To simplify the problem, the semiconductor electrons are divided into two groups, that of valence and core electrons. Core electrons are defined as those in filled orbitals, for silicon these are the $1s^{2}$, $2s^{2}$ and $2p^{6}$ electron configurations. Due to the proximity of core electrons with the nuclei, they are grouped with the nucleus and referred to as ion cores. Valence electrons (incomplete orbitals) for silicon are then in the $3s$ and $3p$ orbitals. With this approximation, indices $j$ and $j\prime$ represent the ion cores while $i$ and $i\prime$ give the valence electrons.

\subsubsection{Born-Oppenheimer or Adiabatic Approximation}

Another approximation is to assume that the electronic motion and nuclear motion within the solid can be separated. This will lead to a molecular wave function given in terms of electron and nuclear positions. This is also known as the adiabatic theorem as stated by Born and Fock in 1928: A physical system remains in its instantaneous eigenstate if a given perturbation is acting on it slowly enough and if there is a gap between the eigenvalue and the rest of the Hamiltonian's spectrum \cite{born1928}. 

For example, consider a Hamiltonian which is separable into two terms with coordinates $q_{1,2}$:
\begin{equation}
    H=H_{1}\left(q_{1}\right) + H_{2}\left(q_{2}\right)
    \label{eq:hamiltonian_seperable12}
\end{equation}
And the Schr{\"o}dinger equation:
\begin{equation}
    H\psi\left(q_{1},q_{2}\right) = E\psi\left(q_{1},q_{2}\right)
    \label{eq:schrodinger_seperable}
\end{equation}

In this case, a wavefunction of the form $\psi\left(q_{1},q_{2}\right) = \psi_{1}\left(q_{1}\right)\psi_{2}\left(q_{2}\right)$ is taken, where $\psi_{1,2}\left(q_{1,2}\right)$ are eigenfunctions of $H_{1,2}$ with the eigenvalues $E_{1,2}$. Then, the eigenfunctions and eigenvalues of the Hamiltonian are given by solving the time independent Schr{\"o}dinger equation as written in the following form:

\begin{equation}
    \left[T_{N} + T_{e} + V_{ee}\left(r\right) + V_{NN}\left(R\right) + V_{eN}\left(r,R\right)\right]\Psi\left(r,R\right) = E\Psi\left(r,R\right)
    \label{eq:TISE_hartree_fock}
\end{equation}

Where $T_{N}$ is the kinetic energy of the nuclei, $T_{e}$ is the kinetic energy of the electrons and $V_{NN},V_{eN},V_{ee}$ are the nucleus-nucleus, nuclei-electron and electron-electron interactions respectively. These terms are 

The first approximation is then introduced by recognising the separation of time scales between electronic and nuclear motion. This is due to the electrons being lighter than nuclei by a factor of approximately 2000: $m_{e}\approx9.11\times10^{-31}\si{\kilogram}, \quad m_{n,p}\approx1.67\times10^{-27}\si{\kilogram}$. After application of this condition, the Hamiltonian and Schr{\"o}dinger equation are reduced down to:

\begin{equation}
    H_{el} = T_{e}\left(r\right) + V_{eN}\left(r,R\right) + V_{NN}\left(R\right) + V_{ee}\left(r\right)
    \label{eq:hamiltonian_fixed_nuclear}
\end{equation}
\begin{equation}
    H_{el}\phi_{e}\left(r,R\right) = E_{el}\phi_{e}\left(r,R\right)
    \label{eq:clamped_nuclear_schrodinger}
\end{equation}

Which is known as the "clamped nuclei" Schr{\"o}dinger equation. $V_{NN}\left(R\right)$ may also be neglected as $R$ is a parameter, making $V_{NN}\left(R\right)$ a constant. Without $V_{NN}\left(R\right)$:
\begin{equation}
    H_{e} = T_{e}\left(r\right) + V_{eN}\left(r,R\right) + V_{ee}\left(r\right)
    \label{eq:electronic_hamiltonian_clamped}
\end{equation}
\begin{equation}
    H_{e}\Phi_{e}\left(r,R\right) = E_{e}\Phi_{e}\left(r,R\right)
    \label{eq:electronic_hamiltonian_clamped_2}
\end{equation}

With the subscript $e$ to distinguish the Hamiltonian and energy from the case with $V_{NN}$. Next, consider the original Hamiltonian for a crystal structure \ref{eq:hamiltonian_of_crystal_structure} with a wavefunction of form $\Psi\left(r,R\right) = \phi_{e}\left(r,R\right)\phi_{N}\left(R\right)$:
%reference issue to the hamiltonian function I don't fucking know why but I'm not fixing it right now, you deal with it future me you're so smart and capable HAHAHA jk
\begin{equation}
    H\phi_{e}\left(r,R\right)\phi_{N}\left(R\right) = E_{tot}\phi_{e}\left(r,R\right)\phi_{N}\left(R\right)
    \label{eq:hamiltonian_with_wavefunction_inserted}
\end{equation}
\begin{equation}
    \left[ T_{N}\left(R\right) + T_{e}\left(r\right) + V_{eN}\left(r,R\right) + V_{NN}\left(R\right) \right] = E_{tot}\phi_{e}\left(r,R\right)\phi_{N}\left(R\right)
    \label{eq:hamiltonian_with_wavefunction_inserted_full}
\end{equation}

As $T_{e}$ is not dependent upon $R$:
\begin{equation}
    T_{e}\phi_{e}\left(r,R\right)\phi_{N}\left(R\right) = \phi_{N}\left(R\right)T_{e}\phi_{e}\left(r,R\right)
    \label{eq:Te_dependence}
\end{equation}
But this does not mean that it is possible to assume the following:
\begin{equation}
    T_{e}\phi_{e}\left(r,R\right)\phi_{N}\left(R\right) = \phi_{N}\left(R\right)T_{e}\phi_{e}\left(r,R\right)
    \label{eq:Te_dependence}
\end{equation}
%https://web.northeastern.edu/afeiguin/phys5870/phys5870/node9.html#:~:text=The%20Born%2DOppenheimer%20Approximation%20is,electron%20positions%20and%20nuclear%20positions. for continuation, if I so want to do that..... probably not needed.
%holy fuck I need to read one of these introductory textbooks and go through some lectures for this too god fuckin damnit hahahahahaha losing my mind trying to figure out what the fuck is going on with these hamiltonians and eigenshits


\subsection{Brillouin Zones}
The one dimensional Schr{\"o}dinger equation as simplified by the prior approximations is given by:
\begin{equation}
    H_{1e}\Phi_{n}\left(r\right)\Phi_{n}\left(r\right) = \left(\frac{p^{2}}{2m}+V\left(r\right)\right)\Phi_{n}\left(r\right) = E_{n}\Phi_{n}\left(r\right)
    \label{eq:schrodinger_1d_equation_motion_of_electrons}
\end{equation}
Where $H_{1e}$ represents the one electron Hamiltonian and $\Phi_{n}$, $E_{n}$ are the wavefunction and 
energy of an electron in an eigenstate $n$.

To calculate the electronic energies $E_{n}$, first the one electron potential $V\left(r\right)$ must be determined. This can be done from first principles with atomic numbers and positions, or with a semi-empirical approach. In the latter options, the electron potential is expressed with parameters that are determined by experimental data fitting. With this, it is still necessary to utilise the crystal symmetry in finding solutions of the Schr{\"o}dinger equation.

All crystal structures possess translational symmetry, and most also have rotational or reflection symmetries. The majority of semiconductors have a high degree of rotational symmetry which allows for a significant simplification of the electronic states within these solids. This simplification is known as band theory.

\subsubsection{Bloch Function}
A particle moving in a periodic potential has a wavefunction which can be expressed in a so called Bloch function. To define this function, we use the one dimensional Schr{\"o}dinger equation \ref{eq:schrodinger_1d_equation_motion_of_electrons} and assume that the electron potential $V\left(x\right)$ is periodic with a translational period equivalent to $R$. Then, a translation operator $T_{R}$ can be defined as:
\begin{equation}
    T_{R}f\left(x\right) = f\left(x+R\right)
    \label{eq:translation_operator}
\end{equation}

Then, the Bloch function $\Phi_{k}\left(x\right)$:
\begin{equation}
    \Phi_{k}\left(x\right) = e^{ikx}u_{k}\left(x\right)
    \label{eq:bloch_function}
\end{equation}
Has a periodic function $u_{k}\left(x\right)$ with the same periodicity as $V$, so that $u_{k}\left(x + nR\right)  = u_{k}\left(x\right)$ for all integers $n$. If $\Phi_{k}\left(x\right)$ is then taken with the product of $e^{ikR}$ it will represent a plane wave with an amplitude that is modulated by the periodic function $u_{k}\left(x\right)$. Hence, $\Phi_{k}\left(x\right)$ is known as a Bloch function. When the position $x$ changes to $x+R$, then the Bloch function becomes:
\begin{equation}
    T_{R}\Phi_{k}\left(x\right) = \Phi_{k}\left(x + R\right) = e^{ikR}\Phi_{k}\left(x\right)
    \label{eq:bloch_function_plusR}
\end{equation}
Then, $\Phi_{k}\left(x\right)$ is an eigenfunction of $T_{R}$ with eigenvalue $e^{ikR}$. The Hamiltonian $H_{1e}$ is invariant under translation by $R$, and $H_{1e}$ commutes with $T_{R}$. Hence, the eigenfunctions of $H_{1e}$ can also be expressed as the eigenfunctions of $T_{R}$. So, the eigenfunction $\Phi\left(x\right)$ of $H-{1e}$ can be written as a sum of Bloch functions:
\begin{equation}
    \Phi\left(x\right) = \sum_{k}A_{k}\Phi_{k}\left(x\right) = \sum_{k}A_{k}e^{ikx}u_{k}\left(x\right)
    \label{eq:sum_of_bloch_functions}
\end{equation}

Where $A_{k}$ is a constant. Therefore the one electron wavefunctions can be indexed by constant k's, given by the wave vectors of the plane waves. The electronic band structure of a given crystal can then be represented by a plot of the electron energies from the one dimensional Schr{\"o}dinger equation \ref{eq:schrodinger_1d_equation_motion_of_electrons} against the wave vectors $k$. This $k$ plot is referred to as $k$ space, or the momentum space, due to the quantum mechanical link between wave vectors and momentum of particles. 

An extended zone scheme is where $k$ is varied over all possible values. However, it is possible to reduce the range of the wave vectors by considering the translation symmetry of the crystal. From the definition of the Bloch function \ref{eq:bloch_function}, it can be seen that the values $k$ for indexing the wave function are not unique. Hence, $k$ and $k + \left(\frac{2n\pi}{R}\right)$ where n is any integer satisfies the Bloch function. Another way to choose k is in replacing $k$ with $k\prime = k - \left(\frac{2n\pi}{R}\right)$, with an integer n that limits $k\prime$ within the interval of $\left[\frac{-\pi}{R},\frac{\pi}{R}\right]$. This interval defines the first Brillouin zone and is also known as the reduced zone scheme. By using the reduced zone scheme, wave functions are indexed by the band index $n$ and the wave vector $k$ is restricted to the first Brillouin zone. Naturally, this definition can be expanded into three dimensions to give a general result for real semiconductors.

In figure (figure I need to plot. matplotlib of the equation for a nice simple output similar to that of yu or sze perhaps?) the band structures with an extended and reduced zone scheme for an electron where $V \rightarrow 0$ is shown, which represents the nearly free electron within a one dimensional lattice of lattice constant a. The reduced zone scheme is used to display the band structures more compactly. Also, electrons which make a transition of state under a translationally invariant operator will conserve $k$ in the reduced zone scheme process.

Generalised into three dimensions and some plots to display the Brillouin zones in 3d.

Primitive lattice vectors:

\begin{equation*}
    \begin{aligned}
        a_{1} = \left(0, \frac{a}{2}, \frac{a}{2}\right) \\a_{2} = \left(\frac{a}{2}, 0, \frac{a}{2}\right) \\a_{3} = \left(\frac{a}{2}, \frac{a}{2}, 0\right)
    \end{aligned}
    \label{eq:primitive_lattice_vectors}
\end{equation*}

Where a represents the side length of the crystallographic unit cell, or the smallest possible cube of the fcc lattice. 

For the direct lattice representation it is also possible to define a reciprocal lattice in similar terms with the relationship:
\begin{equation*}
    b_{i}=2\pi\frac{\left(a_{j}\times a_{k}\right)}{\left(a_{1}\times a_{2}\right)\cdot a_{3}}
    \label{eq:reciprocal_lattice_vectors}
\end{equation*}

In which case $i$, $j$ and $k$ give the cyclic permutation of all indicies and $\left(a_{1}\times a_{2}\right)\cdot a_{3}$ gives the volume of the primitive cell. The resulting reciprocal lattice from translation of the real lattice points is used to represent the wave vector $k$ as a point in reciprocal lattice space. When we bring the Brillouin zones into the reciprocal space, the first zone is defined by the smallest polyhedron that is confined by planes which bisect the reciprocal lattice vectors. The resulting symmetry of Brillouin zones in the reciprocal space are determined by the symmetry of the crystal lattice, with an fcc lattice forming a body-centred cubic (bcc) lattice. The resulting primitive reciprocal lattice vectors are calculated from equation \ref{eq:reciprocal_lattice_vectors}:

\begin{equation*}
    \begin{aligned}
        b_{1} = \left(\frac{2\pi}{a}\right)\left(-1,1,1\right) \\
        b_{2} = \left(\frac{2\pi}{a}\right)\left(1,-1,1\right) \\
        b_{3} = \left(\frac{2\pi}{a}\right)\left(1,1,-1\right) \\
    \end{aligned}
    \label{eq:reciprocal_primitive_lattice_vectors}
\end{equation*}

\paragraph{High Symmetry Points}
The Brillouin zone of a fcc lattice is unchanged by various rotations and is invariant under reflection through certain planes that contain the centre of the cube. Any operation such as this is known as a symmetry operation of the Brillouin zone and results from the symmetry of the direct lattice or real space crystal lattice. Points of high symmetry are by convention, Greek letters when inside the Brillouin zone and Roman letters when on the surfaces. The centre of the Brillouin zone is labelled $\Gamma$. In the three high symmetry directions \hkl[1 0 0], \hkl[1 1 1] and \hkl[1 1 0]:
\begin{equation}
\begin{aligned}
    \relax[1 0 0] = \Gamma \rightarrow \Delta \rightarrow X \\
    [1 1 1] = \Gamma \rightarrow \Lambda \rightarrow L\\
    [1 1 0] = \Gamma \rightarrow \Sigma \rightarrow K    
\end{aligned}
\end{equation}

This has two important impacts on the electronic band structure. 

The first point is that if two wave vectors within the Brillouin zone have a transformation through a symmetry operation of the Brillouin zone, the electronic energies at these vectors must then be identical and the points in reciprocal lattice space are equivalent. 

Secondly, wave functions can be expressed so that they have definite transformations within symmetry operations of the crystal, which results in a symmetric wave function.

The result of the first of these two points is that any of the eight faces of the fcc Brillouin zone are equivalent, with the L point being identical on all eight planes. To calculate the electronic band structure of all eight faces then, only one needs to be calculated. Symmetric or symmetrized wave functions are responsible for the standard notation of wave functions for electrons in atoms, which are labelled based on their transformation properties under rotation with the labels $s$, $p$, $d$, etc. The $s$ wave function is unchanged by any rotation while the $p$ wave functions are triply degenerate and will transform under rotation. $d$ wave functions will transform in the same way that five components of a symmetric traceless second rank tensor would transform, and so on. As a result of this definition of the wave functions, it is possible to demonstrate that certain elements of operators will vanish, leading to selection rules for the wave functions.

\subsection{Diamond Structure Symmetry}

As mentioned previously, the diamond structure consists of two interpenetrating fcc lattices. At each lattice site, there will be two identical atoms that displace in relation to each other by a quarter of the body diagonal in the \hkl[111] direction. The primitive cell that this defines is repeated at every lattice site. When the origin is chosen to be the midpoint of the primitive cell, the crystal structure is invariant under inversion. With an origin located at an atomic site, a translation along the vector $\left(\frac{a}{4}\right)[1,1,1]$ is required for invariance under inversion. The carbon atoms in the \hkl[111] direction are shown in figure (simple drawing of atoms in 111 plane). The resulting space group of the diamond structure is nonsymmorphic, with three glide planes. In Sch{\"o}nflies notation this is denoted as $O_{h}^{7}$ and in international notation, $Fd3m$.

\subsection{Transformations}
Symmetry operations are represented with a transformation matrix. For example, consider a three-fold rotation $\left(xyz\right)\rightarrow\left(zxy\right)$ about the \hkl[111] axis, which transforms the axes $x$, $y$ and $z$ into $x\prime$, $y\prime$ and $z\prime$ with $x\prime=z$, $y\prime=x$ and $z\prime=y$. When represented with a transformation matrix:
\begin{equation}
    \begin{pmatrix}
    0 & 0 & 1\\
    1 & 0 & 0\\
    0 & 1 & 0
    \end{pmatrix}
    \label{eq:three_fold_rotation_111}
\end{equation}

The four-fold transformation $\left(xyz\right)\rightarrow\left(xz\Bar{y}\right)$ about the $x$ axis is given by $x\prime=x$, $y\prime=z$ and $z\prime=-y$ or:
\begin{equation}
    \begin{pmatrix}
    1 & 0 & 0\\
    0 & 0 & 1\\
    0 & -1 & 0
    \end{pmatrix}
    \label{eq:four_fold_roatation_100}
\end{equation}


\section{Fluorescence}
\subsection{Diamond Spectrum}
\subsection{Laser Induced Colour Centres}

\part{Graphitisation of Diamond} % Chapter title

\label{ch:laser} % For referencing the chapter elsewhere, use \autoref{ch:examples} 

%----------------------------------------------------------------------------------------

The roots of laser graphitisation in diamond can be traced back to 1996, with a seminal paper by Davies et al. demonstrating that the usage of tightly focused femtosecond laser pulses could be used to permanently modify the optical properties of a small volume within the transparent substrate \cite{davies:1996}. With careful tuning of the irradiation conditions, it is possible to induce a localised refractive index increase within the focal volume, allowing for the writing of optical waveguides with only a translation of the substrate. 

The advantages of this technique can also be seen through the lens of diamond device processing and graphitisation rather than the tuning of optical properties, though a significant portion of the current industrial applications for laser graphitisation within diamond are in writing serial numbers and other optical markers for jewellers to identify lab-grown diamond:

\begin{itemize}
	\item Laser fabrication is a direct, maskless technique. Without the need for complex clean room facilities it is possible to create surface tracks or buried wires within the diamond. This allows for the rapid prototyping and refinement of a small number of devices, without the added need of dedicated photolithographic masks.
	\item This technique is highly flexible, as different irradiation parameters such as wavelength, pulse energy, translation speed, focusing conditions and repetition rate can be used to calibrate the laser for different materials like silicon carbide, diamond and other crystal structures. This also allows for more specific control over the graphitic content which is formed within the diamond substrate, with differing conductivities for various sp$^{2}$ and sp$^{3}$ content in the formed wires.
	\item With the usage of adaptive optics it is possible to create fully three-dimensional graphitic wires within the diamond substrate, of various thicknesses and conductivities at arbitrary depths. The degrees of freedom offered by this technique allow for geometries that are impossible or highly difficult with standard fabrication techniques, which may be an important factor in the design and manufacturing of certain device structures. 
\end{itemize}

\clearpage
%----------------------------------------------------------------------------------------

\begin{figure}
	\centering
	\includegraphics[width=\linewidth]{"gfx/Phase diagram/HDphase"}
	\caption{The phase diagram of carbon to 50 \si{\giga\pascal}, which corresponds to approximately a 1500 \si{\kilo\metre} depth within the Earth. Original data sourced from \cite{blank:2018}.}
	\label{fig:phase}
\end{figure}

The fundamental process of writing graphitic regions within diamond is governed by the crystal structure metastability. At room temperature and under atmospheric pressure the stable phase of carbon is that of graphite, as shown in figure \ref{fig:phase}, hence when the diamond lattice damage density exceeds a critical value, it will form graphite. This can also be viewed as a sufficient number of diamond sp$^{3}$ bonds being broken, then reforming into the stable graphitic phase with sp$^{2}$ bonds. One simple way to cause this damage to a diamond lattice is to essentially burn it, or otherwise raise the temperature of the sample. A diamond lattice will tend to rearrange globally when heated to $T_{G} \approx 2000$ \si{\kelvin}, which is called the process of thermal graphitisation \cite{davies:1972}. However, it has been observed that in cases where a laser pulse does not generate this critical temperature, graphitisation may still occur. 

\section{Ion Implantation}
Another way to break the diamond sp$^{3}$ bonds is to bombard the diamond lattice with ions of a sufficiently high kinetic energy to also break the sp$^{3}$ bonds. Ion implantation was first investigated in the 70s by Vavilov et al \cite{vavilov:1973}, and resulted in a series of studies examining the electrical properties of the resultant amorphous carbon. Hauser et al. demonstrated the similarities of sputtered graphite to that of the ion implanted diamond, with a high conductivity of $\approx10^{-2}$ \si{\per\ohm\per\centi\metre} and also concluded that the resulting hardness of the implanted regions was an intermediate between that of silicon and diamond \cite{hauser:1976,hauser:1977}. With the proper combinations of implanting ions, ion energies, dosages and post implantation annealing, diamond based heterostructure can be formed which have layers of insulating, semi-conducting, luminescent and fully conductive regions \cite{gippius:1999,prins:1983,prins:1985}. 

One more specific case occurs in the case of implanting boron ions in polycrystalline diamond, where a "percolative threshold" fluence exists at which a conductive path of sp$^{3}$ bonded defects is formed in the diamond structure and a variable range hopping conduction mechanism is introduced. At a higher "amorphisation threshold", sp$^{2}$ bonded defects are formed, leading to permanent graphitisation in the implanted areas after annealing \cite{fontaine:1995}.

\section{Laser Graphitisation}
In the case of laser graphitisation, the mechanism of localised heating through the non-linear absorption of laser pulses requires some clarification, especially since diamond is generally regarded as a "thermal short" with its thermal conductivity of up to 2200-2500 \si{\watt\per\metre\per\kelvin} \cite{graebner:1995}. As presented in an article by Sundaram and Mazur in 2002, while intense laser pulses from a nanosecond pulsed laser within silicon is a thermally driven process, experimentally observed ultrashort femtosecond laser pulse effects cannot be explained with merely a thermal process \cite{sundaram:2002}. Instead, the femtosecond pulses drive a large fraction of the valence electrons ($>10\%$) into the conduction band, modifying the interatomic forces and destabilising the crystal lattice. A model presented by Kononenko et al. in 2015 relies primarily upon the expansion of graphitic inclusions through so called photographitisation, and argues that the laser induced direct creation of graphitic nucleation sites within diamond is unlikely \cite{kononenko:2015}. Instead, graphite inclusion defects within the diamond structure are the source, with inclusions forming the seeds of growing and overlapping graphitic regions. This growth of inclusions is then a two step process, with a distinct mechanism for each stage. Similar work presented by Bennington et al. in 2009 on single crystal (111) oriented HPHT diamonds concurs with this model, observing that the generated graphite was likely seeded by defects within the crystal lattice \cite{bennington:2009}.

\subsection{Photographitisation}
Under femtosecond irradiation, it has been observed that the transition from diamond to graphite is not one driven by a thermal process, but instead it is one driven by an electron-hole plasma that is generated by the laser. Pulsed X-ray diffraction experiments in germanium, silicon, GaAs and InSb observed this same process in 1999 and 2001 with the laser-induced promotion of a large fraction of valence electrons (more than $10$\%) and with a low "melting time" on the order of femtoseconds \cite{siders:1999,rousse:2001}. Once excited, these electrons strongly interact with one another and reach thermal equilibrium in under 10 \si{\femto\second} \cite{elsaesser:1991}.
Through the process of photoionisation, light is absorbed by the substrate and photon energy promotes electrons from bonding orbitals in the valence band to the conduction band, in unbound or even antibonding states. This absorption within diamond is a multiphoton process, depending upon the lasing wavelength. For $800$ \si{\nano\metre}, there is a four or five photon transition, while $400\text{ and }266$ \si{\nano\metre} wavelengths have an indirect and two photon transition respectively \cite{preuss:1995}. The direct band gap in intrinsic diamond for these transitions is $7.3$ \si{\electronvolt}. However, the presence of substitutional nitrogen or phosphorous will introduce donor like centres with a thermal ionisation energy of $1.7$ \si{\electronvolt} and $0.6$ \si{\electronvolt} respectively, complicating the band structure. The electron-hole density within diamond when illuminated by Kononenko et al. with a \ce{TiAl2O3} laser of wavelength $800$ \si{\nano\metre} at fluencies of under $10$ \si{\joule\per\centi\metre\squared} reached up to $10^{21}$ \si{\per\centi\metre\cubed}. The experimental data from this work is plotted in figure \ref{fig:electronholes}.

\begin{figure}
	\centering
	\includegraphics[width=\linewidth]{"gfx/Electron hole pair diagram/HDelectronholes"}
	\caption{The concentration of electron hole pairs in single crystal diamond just below the crystal surface, as measured by Kononenko et al \cite{kononenko:2014}.}
	\label{fig:electronholes}
\end{figure}

In the case of a perfect diamond lattice, these generated electron hole pairs would eventually relax and reform the \ce{C-C} bonds of sp$^{3}$ orbitals. The diamond lattice is highly rigid, which prevents the deformation of bonds via electron hole pairs in contrast to other wide band gap semiconductors \cite{martin:1997}. However, when these electron hole pairs are generated close to the edge of a graphite inclusion or other such graphitic region, then the chemical bonds making up the diamond lattice experience a significant distortion due to local Coulomb forces. Hence when there is a significant density of electron hole pairs, the \ce{C-C} bonds at the interface of diamond and graphite have a chance to change in hybridisation from sp$^{3}$ to sp$^{2}$ orbitals, permanently breaking the diamond lattice and expanding the graphitic region. The exact ratio of sp$^{2}$/sp$^{3}$ within an amorphous carbon structure such as that formed under photographitisation may vary in regions, but areas of less graphitic content may have a ratio approaching 3 based on previous XPS/XAES measurements of amorphous carbon as formed with ion bombardment \cite{lascovich:1991}. Under subsequent pulses of laser irradiation, this process will slowly expand the graphitic inclusion, layer by layer.

\subsubsection{Density Functional Theory}
Through the usage of DFT, atomistic calculations examining the transition from diamond to graphite have estimated that the graphitisation due to photoionisation will complete within $100$ \si{\femto\second} \cite{jeschke:1999} to $200$ \si{\femto\second} \cite{wang:2000}. In particular, the work by Wang et al in 2000 examined the particular case of laser induced graphitisation on the (111) surface. This is particularly relevant for devices written on phosphorous doped, n-type diamond, as the (111) surface provides the highest dopant concentration and also most active carrier concentration when compared to other growth facets. Their conclusions can be summarised as:

\begin{itemize}
	\item When the overall diamond temperature reaches 2700 \si{\kelvin} the surface will rapidly graphitise.
	\item Graphitisation occurs vertically for the simulations of longer pulse (nanosecond) durations, with graphite-like regions penetrating down through the surface layers into the bulk.
	\item Graphite sheets are formed in the case of femtosecond pulse durations, with a reduced time for complete graphitisation at higher effective electron temperatures.
	\item The energy barrier to graphitisation is lower in the case of higher electron temperatures, also corresponding to a reduced time required for complete graphitisation.
\end{itemize}

More recent studies examining the stability of diamond and graphite bonds with larger supercells provide further confirmation of the energy barrier between the two carbon hybridisations \cite{popov:2019,grochala:2014}. A first-principle study by F. Mauri in 1995 explored the electron lattice interactions, concluding that valence excitons are highly likely to bind together when in a high density electron-hole plasma \cite{mauri:1995}. Self-trapping does not occur in the case of isolated valence excitons, but it is predicted that self trapping should occur with valence biexcitons. This biexciton trapping will cause a distortion, involving the breaking of a bond perpendicular to the (111) direction. As graphite is semi-metallic in nature, electron-hole pairs migrating from the excited diamond regions to that of a graphite seed will rapidly multiply through impact ionisation. Impact will also reduce their energy below the diamond band gap, trapping such pairs and collecting them within the graphtitic region. Graphite's $\pi$ orbitals are able to hold a large number of electron-hole pairs without changing the covalent bond network, lending a greater stability to the graphite region than the diamond in conditions of high electron-hole pairs in addition to the refractory nature of graphite \cite{bennington:2009}.

\begin{figure}
	\centering
	\includegraphics[width=\linewidth]{"gfx/Fluence vs number of laser shots/HDfluencevsshots"}
	\caption{The number of laser shots required for visible damage of the diamond surface as a function of laser fluence, as measured by Kononenko et al \cite{kononenko:2008}.}
	\label{fig:fluencevsshots}
\end{figure}

\subsection{Thermal Graphitisation}
As these graphite seeds grow due to the photostimulated lattice rearrangement of carbon atoms adjacent to the graphite, the amount of light that is absorbed due to the change in the optical properties will increase. Also, when stimulated by light, the electron subsystem of graphite will take longer to relax than its sp$^{3}$ counterpart from the excited $\pi$-band electron population. This effect slows the heating of graphitic regions on a scale of $\tau_{e-ph}\approx1$ \si{\pico\second}, where $\tau_{e-ph}$ is the electron-phonon thermalisation \cite{seibert:1990}. Diamond has a remarkably high thermal diffusivity of $\chi_{d}\approx10$ \si{\centi\metre\squared\per\second} \cite{tokmakoff:1993}. This is high enough that the process of thermalisation can overlap in time with the heat spreading from graphite seeds into the diamond lattice. Hence, the temperature of these diamond seeds as induced by laser stimulation will strongly depend upon the seed size, which can be expressed in the following way:

\begin{equation}
\Delta T \approx\sigma_{a}F/l_{D}^{3}/c_{d}
\label{eq:deltaT}
\end{equation}
\begin{equation}
\sigma_{a} \approx \frac{2\pi}{\lambda}Im\left(4\pi b^{3}\frac{\epsilon_{g}-\epsilon_{d}}{\epsilon_{g}+2\epsilon_{d}}\right)
\label{eq:sigma laser}
\end{equation}
\begin{equation}
l_{D} = \left(4\chi_{d}\tau_{e-ph}+b^{2}\right)^{1/2}
\label{eq:ld laser}
\end{equation}

Where $\sigma_{a}$ is the absorption cross section, $\epsilon_{g}=6.0825+10.673i$ \cite{djurisic:1999} is the permittivity of graphite and $\epsilon_{d}=5.76$ \cite{phillip:1964} is the permittivity of diamond, $F$ is the laser fluence, $b$ is the graphite seed radius and $c_{d}=1.75$ \si{\joule\per\centi\metre\cubed\per\kelvin} is the specific heat capacity of diamond \cite{prelas:1997}. The energy absorbed by the graphite seed will dissipate in a volume $l_{D}^{3}$, defined by the current radius of the seed and the heat diffusion length within diamond. Heat diffusion length is defined by the thermalisation time of electron-phonons  $\tau_{e-ph}\approx1$ \si{\pico\second}. $\Delta T\left(b\right)$ as calculated with equations \ref{eq:deltaT}, \ref{eq:sigma laser}, \ref{eq:ld laser} at a fluence of $0.75$, $1.5$ and $3$ \si{\joule\per\centi\metre\squared} and $\tau_{e-ph}=1$ \si{\pico\second} is depicted in figure \ref{fig:laser-heating}.

\begin{figure}
	\centering
	\includegraphics[width=\linewidth]{"gfx/Seed temperature/laser heating"}
	\caption{The estimated laser-induced heating according to the model presented by Kononenko et al, represented by equations \ref{eq:deltaT}, \ref{eq:sigma laser}, \ref{eq:ld laser} at fluencies from $0.75$ \si{\joule\per\centi\metre\squared} to $3$ \si{\joule\per\centi\metre\squared}.}
	\label{fig:laser-heating}
\end{figure}

The critical seed radius at which thermal graphitisation becomes possible is marked as $b_{c1,2,3}$ for the three different fluencies respectively. It can be noted that at a lower laser fluence, a larger critical radius is calculated as the fluencies $0.75$, $1.5$ and $3$ \si{\joule\per\centi\metre\squared} resulted in a calculated critical radius of $19$, $15$ and $12$ \si{\nano\metre} respectively. This can be compared to figure \ref*{fig:fluencevsshots}, in which the number of laser pulse "shots" required for experimentally observed graphitisation is plotted against the laser fluence. This comparison is represented in figure \ref{fig:comparisonshots}. It is possible to infer a few points from this comparison, as the natural conclusion of lower fluence laser pulses requiring a larger critical radius also demands a larger number of laser shots to observe experimental graphite growth. This fits the model well, as the photographitisation step of adding layers to the graphitic defects will take far longer than the thermal graphitisation step. Indeed, this manifests as the requirement for an order of magnitude more laser pulses at merely half the laser fluence. It is hoped that under a certain fluence, the critical radius will be reduced to the scale of sub-microns and such will never be able to reach the thermal graphitisation stage of growth. Hence, the only graphitisation mechanism will be that of photographitisation at very low laser fluencies.

\begin{figure}
	\centering
	\includegraphics[width=\linewidth]{"gfx/Comparison of fluence and shots/comparisonshots"}
	\caption{A comparison of the laser fluencies used in the calculation of critical seed radii and the observed number of laser shots required for the process of graphitisation. The lower fluence requires a larger seed radius for thermal graphitisation and hence requires many more laser pulses for the photographitisation growth.}
	\label{fig:comparisonshots}
\end{figure}

\section{Graphite}
Graphite is typically manufactured by pressing or extruding a mixture of carbon containing materials such as pitch, coal, petrochemicals or other high molecular weight hydrocarbon, followed by a super-heating bake at anywhere from 2000 \si{\degreeCelsius} to more than 4000 \si{\degreeCelsius} to crystallise the amorphous carbon precursors. 

This heating is typically a multi-stage process, with the first step requiring a long, complex heating in vacuum of the carbon containing material to convert it into coke. This process is known as destructive distillation, with a "good coke" containing a very high carbon content and few if any impurities. The coke is then calcinated, crushed and sieved to obtain a certain distribution of particle sizes. These particles are then bound with a binding substance such as coal tar pitch, petroleum pitch or a synthetic resin and extruded or moulded to form the graphite products. Next, a carbonisation bake at around $1000\text{--}1200\si{\degreeCelsius}$ will thermally decompose the binder into elementary carbon and other volatile components, binding the powdered graphite together. The resulting volume of the formed carbon is lower than that of the binder, due to the formation of pores, with a relative volume or porosity depending upon the binder quantity.

Finally, the shaped and carbonised parts can be baked in the absence of oxygen, or in vacuum, at very high temperatures to induce crystallisation of the amorphous precursor carbon. Typical operating temperatures are in the $2500\text{--}3000\si{\degreeCelsius}$ range, which will also purify the graphite parts due to the vaporisation of impurities such as any remaining binder residue, oxides or gases. This is usually achieved with either induction based heating, or passing electric currents directly through the parts to utilise Joule heating.

A variety of precursors can be used, with gilsocarbon being a common choice for usage in nuclear power plants \cite{liu2017}.

It is also possible to form graphite via chemical vapour deposition (CVD) techniques, which is known as pyrolitic graphite. This form of graphite will have a very low porosity, approaching the theoretical density of graphite, with correspondingly improved material properties in most other aspects too.

%Past a critical size, the graphite seeds will be able to reach the critical temperature ($T_{g}\approx 2000$ \si{\kelvin}) required for thermal graphitisation within diamond.

%DEVICES
\subsection{On-State Resistance in Diamond Schottky Barrier Diodes}

\subsubsection{Significance of \( R_{\text{ON}}^{\text{spec}} \) in Diamond SBDs}
The on-state resistance, denoted as \( R_{\text{ON}}^{\text{spec}} \), is a crucial figure of merit (FOM) in assessing the performance of diamond Schottky barrier diodes (SBDs). This parameter is critical in evaluating the efficiency and power handling capabilities of the diode under operational conditions. The specific on-state resistance is typically derived through pulsed I-V measurements or via analytical and numerical analyses. It's important to note that \( R_{\text{ON}}^{\text{spec}} \) provides a means to compare different devices or materials over a range of breakdown voltages (BVs) \cite{donato2019}.

\subsubsection{Trade-offs and Challenges}
One of the key findings from the review is the trade-off in performance that is observed in diamond SBDs, particularly when comparing boron-doped (p-type) diamond devices at room temperature (RT) and high temperatures (HT). At HT, boron-doped diamond shows a more favourable trade-off, demonstrating its potential in high-temperature applications \cite{donato2019}.

However, predicting and optimising the on-state resistance in diamond SBDs poses challenges, particularly due to the incomplete ionisation of dopants in diamond, which is a factor in diamond bulk devices. This incomplete ionisation can lead to higher on-state resistances, thereby affecting the device's efficiency \cite{donato2019}.

\subsubsection{Impact on Switching Losses}
The switching losses in diamond power devices, including SBDs, are an area that requires careful consideration. These losses depend on various parameters such as parasitic capacitances, gate driver characteristics, and circuit elements like parasitic inductances and capacitances. A fair assessment of switching losses requires the consideration of these factors along with constraints like electromagnetic compatibility and electromagnetic interference. However, due to the limited availability and small size of diamond FETs, studies on switching losses in diamond devices have been scarce. It is anticipated that diamond FETs would exhibit low switching losses due to their rapid turn-off capabilities and smaller input capacitance \cite{donato2019}.

\subsubsection{Future Perspectives}
Looking ahead, there is potential for significant advancements in diamond SBDs, particularly with regard to minimsing on-state resistance and optimising switching performances. Advances in doping techniques, growth methods, and device fabrication could enhance the electrical properties of diamond, leading to more efficient and reliable SBDs. Your work on reducing on-state resistance by examining ohmic contacts to n-type diamond could address some of these challenges, contributing to the development of more efficient diamond-based power devices.
